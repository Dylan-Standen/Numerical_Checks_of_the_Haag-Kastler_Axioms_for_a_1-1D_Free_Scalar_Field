\documentclass[11pt]{article} 
\usepackage[left=2.2cm,right=2.2cm,top=2.5cm,bottom=2.5cm]{geometry} 
\usepackage{amsfonts,amssymb,amsthm,amsmath,amscd}
\usepackage{enumerate}
\usepackage{tikz}
\usepackage{physics}
\usepackage{hyperref}
\usepackage{slashed}
\usepackage{mathrsfs}
\usepackage{slashed}
\usepackage{tipa} 
\usepackage{notoccite}
\usepackage{cancel}
\usepackage{bbm}


\theoremstyle{definition}
\newtheorem{definition}{Definition}[section]

\title{Numerical Checks of the Haag–Kastler Axioms for a 1+1D Free Scalar Field}
\author{Dylan Standen\thanks{dylansion38@gmail.com}}
\date{}

\setlength{\parindent}{0pt}

\newcommand{\edth}{\eth}
\newcommand{\edthb}{\bar{\eth}}
\newcommand{\nab}{\nabla}
\newcommand{\eps}{\varepsilon}

\usepackage{graphicx}


\usepackage[capitalise,noabbrev]{cleveref} % 
\usepackage{overpic} 
\usepackage{url} 
\usepackage{comment} 


\newcommand\blfootnote[1]{
  \begingroup
  \renewcommand\thefootnote{}\footnote{#1}%
  \addtocounter{footnote}{-1}%
  \endgroup
}
\bibliographystyle{unsrt}


\numberwithin{equation}{section}


\newtheorem{theorem}{Theorem}
\newtheorem{lemma}{Lemma}
\newtheorem{corollary}{Corollary}
\newtheorem{proposition}{Proposition}
\theoremstyle{definition}
\newtheorem{remark}{Remark}
\newtheorem{example}{Example}
\newtheorem{axiom}{Axiom}

\numberwithin{theorem}{section}
\numberwithin{lemma}{section}
\numberwithin{corollary}{section}
\numberwithin{proposition}{section}
\numberwithin{definition}{section}
\numberwithin{remark}{section}
\numberwithin{axiom}{section}


\newcommand{\vect}[1]{\mathbf{#1}}
\setlength{\parindent}{0cm}
\newcommand{\neigh}[1]{\mathcal{O}_{#1}}

\newcommand{\complex}[1]{L(#1)_\mathbb{C}}

\newcommand{\lie}[1]{L(#1)}

\newcommand{\comrep}[1]{#1_\mathbb{C}}

\newcommand{\dual}[2]{T^*_{#1}#2}

\newcommand{\tang}[2]{T_{#1}#2}

\newcommand{\Hom}[3]{\text{Hom}_{\mathcal{#1}}(#2,#3)}

\newcommand{\ob}[1]{\text{Ob}(\mathcal{#1})}


\newcommand{\set}[2]{\left\{ #1 \hspace{0.15cm} \middle| \hspace{0.15cm} #2 \hspace{0.15cm} \right\}}
\newcommand{\intd}[1]{\int \dd^4 #1}  
\newcommand{\Intd}[2]{\int \dd^#1 #2}
\newcommand{\reals}[1]{\mathbb{R}^{#1}}

\newcommand{\lag}{\mathcal{L}}

\newcommand{\del}[1]{\partial_#1}

\newcommand{\innerprod}[2]{\left< #1 \middle| #2 \right>}

\newcommand{\basis}[1]{\{#1\}}

\newcommand{\lieb}[2]{\left[#1, #2 \right]}

\newcommand{\form}[1]{#1-\text{form}}

\newcommand{\ene}[1]{\omega_\vect{#1}}



\begin{document}
\frenchspacing

\maketitle
\begin{abstract}
   \noindent
    Within this text, we discuss and briefly introduce algebraic quantum field theory by discussing both the Wightman and Haag-Kastler formulations. We then undertake a numerical check of Haag-Kastler axioms by solving the Klein-Gordon equation on a lattice of length $L=Na$, using the finite dimensional Fourier modes expansion for the fields in order to test axioms such as locality using covariance matrices. 
\end{abstract}
\bibliographystyle{unsrt}

\frenchspacing




\tableofcontents

\newpage


\section{Introduction:}

The aim of algebraic quantum field theory is to rigorously formulate and define the fundamental axioms that allow us to construct a rigorous and mathematically reasonable description of quantum field theories. There are two sets of axioms: \textit{Wightman axioms} and the \textit{Haag--Kastler} axioms. They differ only in their levels of abstraction, however, as they try and formally describe the same theory we expect that they must be equivalent. There is, however, some difficulty in transitioning between the two. The Wightman axioms aim to describe how the fields defined on Minkowski space behave under Poincaré transformations and also impose causality. For completeness, we shall include them.

\subsection{Wightman Axioms (briefly):}
Before we present the Wightman axioms, we shall assume the reader is somewhat familiar with distribution theory and group theory; hence, we will avoid giving definitions of things like Schwartz space, tempered distributions, representations, etc. If not, a thorough introduction can be seen here \cite{dist}. As well as this, for definiteness and generality, we state the Wightman axioms in 3+1 dimensions, with $M = \mathbb{R}^{1,3}$ and Lorentzian signature, i.e., regular Minkowski space. In the explicit numerical example later, we specialise to 1+1 dimensions.
\begin{axiom}(Space of States)
    \\
    There exists a physical Hilbert space, $\mathcal{H}$, in which a unitary representation $U: P \rightarrow U(\mathcal{H})$ of the Poincaré group $P = \mathbb{R}^{1,3} \ltimes SO^\uparrow(1,3)$ acts, with $\mathbb{R}^{1,3}$ the additive group, and $SO^\uparrow(1,3)$ the proper Lorentz group. 
\end{axiom}

\begin{axiom}(Spectrum Condition)
    Given the closed light-cone $\bar{\mathcal{V}} = \bar{\mathcal{V}}^+ \cup \bar{\mathcal{V}}^-$ and the generator of translations, $P^\mu$, the joint spectrum of $P^\mu$, $\mu = 0,1\cdots,3$ is concentrated in the closed forward light cone $\bar{\mathcal{V}}^+ =\{p\in\mathbb R^{1,3}\mid p^0\ge 0, \hspace{0.15cm} p^2\ge 0\}$, with $p^2 = p^\mu p_\mu$. Explicitly, 
    \begin{equation*}
        \text{Spec}(P_0, \cdots, P_3) \subset  \bar{\mathcal{V}}^+ \hspace{0.15cm} .
    \end{equation*}
\end{axiom}

\begin{axiom}(Vacuum State)
    \\
   There exists a unique state $\Omega \in \mathcal{H}$, of unit norm, which is invariant under the whole Poincaré group: $U(a,\Lambda)\Omega=\Omega$ for all $(a,\Lambda)$.  
\end{axiom}

\begin{axiom}(Operator Valued Fields)
\\
The quantum field's components $\phi_a$, $a \in I$ (with $I$ some indexing set), are tempered distributions over the Schwartz space $\mathscr{S}(\mathbb{R}^{1,3})$ with a dense subspace $D \subset \mathcal{H}$ as their common domain. The vacuum is contained within $D$ (as well as its scalar multiples) and $D$ is invariant: i.e. 

\begin{align*}
    \phi_a(f)D & \subset D, \hspace{0.25cm} \forall \hspace{0.15cm} f \in \mathscr{S}, a \in I \hspace{0.15cm} , & U(a,\Lambda) D & \subset D, \hspace{0.25cm} \forall \hspace{0.15cm} (a, \Lambda)\in P \hspace{0.15cm} .
\end{align*}
\end{axiom}


It is common to slightly abuse notation and, instead of writing $\phi$ as $\phi(f)$ in terms of a test function $f \in \mathscr{S}$, write $\phi(x)$ with $x \in \mathbb{R}^{1,3}$. This way, we speak about the function $\phi$ at a point in our spacetime, rather than the value of the distribution $\phi(f)$.


\begin{axiom}(Locality) \label{1.5}\\
    Given the field components $\phi_i(x)$ and $\phi_j(y)$, such that their arguments $x$ and $y$ are spacelike separated (if given in terms of test functions $f,g\in \mathscr{S}$ we require their supports be spacelike instead), then, on $D$, 

    \begin{equation*}
        [\phi_i(x), \phi_j(y)] = 0 \hspace{0.15cm} .
    \end{equation*}
\end{axiom}
Note that this is for a bosonic field. A more general statement would include the $\mathbb{Z}_2$ graded commutator.


\begin{axiom} (Covariance) \label{1.6}\\
The components $\phi_i(x)$ transform covariantly under conjugation by $U(a, \Lambda)$, i.e. 

\begin{equation*}
    U(a, \Lambda)\phi_i(x)U^\dagger(a, \Lambda) = \sum_{j}V_{ij}(\Lambda)\phi_j\left(\Lambda^{-1} (x - a)\right) \hspace{0.15cm} ,
\end{equation*}
where $V_{ij}\left(\Lambda\right)$ is a complex or real finite-dimensional matrix representation of the Lorentz group.
\end{axiom}

\begin{axiom} (Completeness)
    The set $\Tilde{D}$ of finite (or infinite in some circumstances) linear combinations of states of the form $\phi_{i_1}(f_1)\cdots \phi_{i_k}(f_k) \Omega$ is dense in $\mathcal{H}$, i.e.

    \begin{equation*}
        \Tilde{D} = \text{Span}\{\phi_{i_1}(f_1)\cdots \phi_{i_k}(f_k) \Omega \hspace{0.15cm} | \hspace{0.15cm} k \in \mathbb{N}, \hspace{0.15cm} f_j \hspace{0.15cm} \in \mathscr{S} \} \subset D \hspace{0.15cm} .
    \end{equation*}
\end{axiom}

As these axioms are not the main focus of this note, we shall not discuss them further, other than to draw comparisons with the Haag-Kastler axioms.\\

As stated, the Haag-Kastler axioms can be seen as a generalisation of the Wightman axioms. As we focus on \textit{nets of algebras} instead of looking at the operator valued distributions, we deem it necessary to introduce what exactly a net is. This will require some baby Category theory, but it will be insightful. 

\subsection{Baby Category Theory:}

The Haa-Kastler axioms can be written naturally in the language of category theory; hence, it will be useful to define and introduce what each term means. This section does not aim to be an extensive look into the subject; thus, an interested reader should find a more thorough review here \cite{}.

\subsubsection{Net of Algebras as a Precosheaf:}

\begin{definition}
    A category, $\mathcal{C}$ (or $\mathbf{C}$; we will use the two interchangeably, purely for aesthetic purposes) is defined to contain:
    \begin{itemize}
        \item A collection of objects, $\text{Ob}(\mathcal{C})$ (i.e. it could be collection of sets, but it could be other objects.)
        \item A collection of Morphisms (or maps), $\Hom{C}{A}{B}$, between $A$ and $B$ for each $A,B \in \text{Ob}(\mathcal{C})$. 
        \item For all $A,B,C \in \text{Ob}(\mathcal{C})$ there exists a \textit{composition} map such that
        \begin{align*}
            \circ: \Hom{C}{A}{B} \times \Hom{C}{B}{C}  & \rightarrow \Hom{C}{A}{C}\\
            (f,g) & \mapsto g\circ f \hspace{0.15cm} .
        \end{align*}
        \item For all $A \in \ob{\mathcal{C}}$, there exists an element $\text{id}_A \in \Hom{C}{A}{A}$, called the \textit{identity} on $A$ such that:

        \begin{align}
            f \circ \text{id}_A = f = \text{id}_A \circ f, \hspace{0.15cm} \forall \hspace{0.15cm} f \in \Hom{C}{A}{A}.
        \end{align}
        \item The map $\circ$ is required to satisfy:
        \begin{itemize}
            \item \textit{Associativity}: for all $f\in \Hom{C}{A}{B}$, $g \in \Hom{C}{B}{C}$, and $h \in \Hom{C}{C}{D}$,

            \begin{equation*}
                h\circ (g \circ f) = (h \circ g) \circ f
            \end{equation*}
            \item \textit{Identity composition}: for all $f \in \Hom{C}{A}{B}$,

            \begin{equation*}
                f \circ \text{id}_A = f = \text{id}_B \circ f \hspace{0.15cm} ,
            \end{equation*}
            where $\text{id}_A \in \Hom{C}{A}{A}, \text{id}_B \in \Hom{C}{B}{B}$. This should hopefully be familiar. For example, in the language of tensor products we could write, given two vector spaces $V_A$, $V_B$ and a linear map $A:V_A \rightarrow V_A$,

            \begin{equation*}
                (\mathbbm{1}_A \otimes \mathbbm{1}_B) \otimes (A \otimes \mathbbm{1}_B) = (A \otimes \mathbbm{1}_B) \otimes (\mathbbm{1}_A \otimes \mathbbm{1}_B) \hspace{0.15cm} .
            \end{equation*}    
        \end{itemize}
     \end{itemize}
\end{definition}

Another object is the \textit{dual} or \textit{opposite} category:

\begin{definition}
    Every category $\mathcal{A}$ has an opposite (or dual) category $\mathcal{A}^{\text{Op}}$, which is defined as the same category as $\mathcal{A}$ but with the maps reversed. This is to say $\ob{\mathcal{A}^{\text{Op}}} = \ob{\mathcal{A}}$, but $\Hom{A^{\text{Ob}}}{A}{B} = \Hom{A}{B}{A}$. Note that composition is also preserved, but with the arguments reversed.
\end{definition}

For our case, it will be important that the morphisms within our category satisfy the following:

\begin{definition}
    A morphism $f \in \Hom{C}{B}{C}$, for $A, B \in \ob{C}$, is said to be a \textit{monomorphism} if

    \begin{equation*}
        \forall \hspace{0.15cm} h_1, h_2 \in \Hom{C}{A}{B},\hspace{0.35cm}  f\circ h_1 = f \circ h_2 \iff h_1 = h_2.
     \end{equation*}
\end{definition}

This requirement essentially generalises injectivity. \\




Now, in our treatment of the Haag-kastler axioms, the category we care about is the collection of open subsets of Minkowski space (of dimension 3+1 in our definitions and 1+1 in our applied numerical example). It will also be useful to define a map between two categories:

\begin{definition}
    Let $\mathcal{A}$ and $\mathcal{B}$ be categories. Then a (covariant) Functor, $F: \mathcal{C} \rightarrow \mathcal{D}$, is the map such that:
    \begin{itemize}
        \item For all $A \in \ob{\mathcal{C}}$
        \begin{equation*}
            A \mapsto F(A) = B \in \ob{\mathcal{B}}.
        \end{equation*}
        \item For all morphisms $f \in \Hom{C}{A}{B}$, with $A,B \in \ob{C}$, 
        \begin{equation*}
            f \mapsto F(f) \in \Hom{D}{F(A)}{F(B)}\hspace{0.15cm}.
        \end{equation*}
        I.e., each morphism in $\mathcal{C}$ is associated with a morphism in $\mathcal{D}$.
    \end{itemize}
    The functor must also satisfy the following axioms:

    \begin{itemize}
        \item Given $f \in \Hom{C}{A}{B}$ and $g \in \Hom{C}{B}{C}$, then

        \begin{equation*}
            F(g \circ_\mathcal{C} f) = F(g) \circ_\mathcal{D} F(f) \in \Hom{D}{F(A)}{F(C)},
        \end{equation*}

        with $\circ_\mathcal{C}, \circ_\mathcal{D}$ representing the composition in $\mathcal{C}$ and $\mathcal{D}$, respectively. The most obvious applied example of this law would is a group homomorphism. Given the groups $(G, \cdot)$ and $(H, *)$, the map $f: G \rightarrow H$ is a group homomorphism iff
        \begin{equation*}
            f(g_1 \cdot g_2) = f(g_1) * f(g_2).
        \end{equation*} 
        \item For all $A \in \ob{A}$,

        \begin{equation*}
            F(\text{id}_A) = \text{id}_{F(A)} \hspace{0.15cm} .
        \end{equation*}
        A clear example of this law would be a representation $\rho: G \rightarrow GL(V)$ of a Lie group $G$ with identity $e$, such that

        \begin{equation*}
            \rho(e) = 1_V ,
        \end{equation*}
        with $1_V$ the identity endomorphism on $V$. In this case, the two categories would be $\mathbf{G}$, the one object category with object $A$, maps $\text{Hom}_{\mathbf{G}}(A,A) \in  G $ (i.e. group elements), with composition given by group multiplication, and the one object category $\mathbf{GL}(V)$ with object $B$, maps $\text{Hom}_{\mathbf{GL}(V)}(B,B)  \in GL(V)$, and composition given by matrix multiplication.
    \end{itemize}
\end{definition}

We may similarly define a \textit{contravariant functor} from the category $\mathcal{A} \rightarrow \mathcal{B}$ as a functor, but with the sources and targets of the morphisms flipped. In other words, it is the functor from $\mathcal{A}^\text{Op} \rightarrow \mathcal{B}$.
Given $f \in \Hom{A}{A}{B}$,

\begin{equation*}
    f \mapsto F(f) \in \Hom{B}{F(B)}{F(A)}\hspace{0.15cm} ,
\end{equation*}
and for $g \in \Hom{A}{B}{C}$

\begin{equation*}
    F(g \circ_{\mathcal{A}}f) = F(f) \circ_\mathcal{B} F(g) \hspace{0.15cm} .
\end{equation*}

If we were to relate this to a physical context, the categories would be \textbf{Man} with smooth manifolds $M, N, \cdots$ as objects and maps associated with smooth maps (in particular, diffeomorphisms), e.g. $f:M \rightarrow N$, and \textbf{VBun} where the objects are vector bundles over $M,N,\cdots$, with the maps given by the vector bundle maps, e.g. $H: E\rightarrow R$, with $E$ a vector bundle over $M$ and $R$ a vector bundle over $N$, which must satisfy

\begin{equation*}
    \pi_R \circ H = f \circ \pi_E \hspace{0.15cm} ,
\end{equation*}

with $\pi_R:R \rightarrow N$ and $\pi_E:E \rightarrow M$. With this, the (tangent) functor would be the map $T: \textbf{Man}\rightarrow \textbf{VBun}$, such that an object $M\mapsto TM$ and a morphism $f\mapsto Tf:TM \rightarrow TN$ (i.e. essentially the pushforward). Similarly, the contravariant functor (cotangent functor) $T^*: \textbf{Man}^{\text{Op}}\rightarrow \textbf{VBun}$ such that $M \rightarrow T^*M$ and $f \mapsto T^*f:T^*N  \rightarrow T^*M$. Notice that this is the pull-back of differential forms.\\


In order to describe a precosheaf we must also introduce the concept of a poset:

\begin{definition}
    A \textit{Poset} is a set $X$ with the binary relation $\leq$, denoted $(P, \leq)$, that is reflexive, antisymmetric, and transitive. Explicitly, is satisfies:

    \begin{itemize}
        \item \textit{Reflexivity}: $\forall \hspace{0.15cm} a \in X, \hspace{0.15cm} a \leq a$.
        \item \textit{Anti-symmetry}: $\forall \hspace{0.15cm} a,b \in X$ if $a \leq b$ and $b \leq a$, then $a = b$.
        \item \textit{Transitivity}: $\forall \hspace{0.15cm} a,b,c \in X$ if $a \leq b$ and $b \leq c$, then $a \leq b$.
    \end{itemize}
\end{definition}

A clear example of this is the reals with the canonical ``less-than-or-equal-to'' relation $\leq$. The above definition introduces a partial ordering on the set (hence the name); essentially, it allows us to determine how a pair of elements is ordered. This is compared with a totally ordered set.


Another concept useful to us is a directed set; 

\begin{definition}
    A directed set $D$ is the poset such that 

    \begin{equation*}
        \forall \hspace{0.15cm} a,b \in D\hspace{0.15cm}, \hspace{0.15cm} \exists \hspace{0.15cm} c \in D \hspace{0.25cm} \text{such that} \hspace{0.15cm} a\leq c \hspace{0.15cm} \text{and} \hspace{0.15cm} b\leq c \hspace{0.15cm} .
    \end{equation*}
\end{definition}

This definition essentially introduces an upper bound within the poset.\\


As stated previously, our category of interest will be the open subsets of Minkowski space; thus, it is necessary to define the following category:

\begin{definition}
    Let $X$ be a topological space, and let $\mathcal{U}(X)$ denote the category whose objects are the open sets of $X$ and whose unique morphism is the inclusion map.
\end{definition}


We may now define a precosheaf:

\begin{definition}
    Let $\mathcal{C}$ be any category. Then, a $\mathcal{C}$-valued precosheaf (presheaf) is the functor $F: \mathcal{U}(X) \rightarrow \mathcal{C}$ ($F: \mathcal{U}(X)^{\text{Op}} \rightarrow \mathcal{C}$).
\end{definition}

Let us now introduce the concept of a net:

\begin{definition}
    Given a directed set $D$ and a set X, a \textit{net} in $X$, denoted by $x_\# = (x_d)_{d \in D}$, is a function $x_\#:D \rightarrow X$.
\end{definition}

This definition applies to our context, where $X$ is a topological space. A net is essentially a generalisation of sequences, i.e., maps from $\mathbb{N} \rightarrow X$. Similar to sequences, we may define limits.\\

How does this link to our formulation? Given Minkowski space, the poset $(\mathcal{U}(M), \subseteq)$ of open sets is naturally a category. Its objects are the open sets in $\mathcal{U}(M)$, and there exists a unique morphism $U_1 \rightarrow U_2$ for $U_1, U_2 \in \mathcal{U}(M)$, given by the inclusion map  

\begin{equation*}
    i:U_1 \hookrightarrow U_2,
\end{equation*}

which we may also denote as $i_{U_1, U_2}$, precisely when $U_1 \subseteq U_2.$ Note that the poset is directed, as the natural upper bound is the union $U_1 \cup U_2$. In the formulation of AQFT, the category to which we wish our functor to map is $\textbf{C*Alg}$, which is the category of unital $C^*$ algebras with unital$^*$-homomorphisms as morphisms. We will not introduce  $C^*$-algebras in this text, as they are a common topic. For the unfamiliar reader, we suggest this text \cite{C^*}. Let $F$ be the functor $F: (\mathcal{U}(M), \subseteq) \rightarrow \textbf{C*Alg}$; then, for each object $U,\in (\mathcal{U}(M), \subseteq)$, 

\begin{equation*}
    U \mapsto F(U) =\mathcal{A}({U}) \in \ob{\mathbf{C^*Alg}} \hspace{0.15cm} .
\end{equation*}

Also, for each $U_1, U_2 \in (\mathcal{U}(M), \subseteq)$, such that $U_1 \subseteq U_2$, 

\begin{equation*}
    F(i_{U_1, U_2}) = \iota_{U_1, U_2}: \mathcal{A}({U_1}) \rightarrow \mathcal{A}({U_2}), 
\end{equation*}

which is also an inclusion. Note that this map is a unital injective $^*$-homomorphism. On our directed set, the identity morphism corresponds trivially to $i_{U,U} = \text{id}_U, \hspace{0.15cm}\forall \hspace{0.15cm} U \in (\mathcal{U}(M), \subseteq)$. Under $F$ we have

\begin{equation*}
    \iota_{U,U} = \text{id}_{\mathcal{A}(U)} \hspace{0.15cm} .
\end{equation*}

As well as this, given $U_1 \subseteq U_2 \subseteq U_3 \hspace{0.15cm} \in (\mathcal{U}(M), \subseteq)$, the composition

\begin{equation*}
    i_{U_2, U_3} \circ_{(\mathcal{U}(M), \subseteq)} i_{U_1, U_2} = i_{U_1, U_3} 
\end{equation*}

maps to

\begin{equation*}
    \iota_{U_2, U_3} \circ_{\mathbf{C^*Alg}} \iota_{U_1, U_2} = \iota_{U_1, U_3} \hspace{0.15cm} .
\end{equation*}

This then implies that this functor is a $C^*$-valued precosheaf. Since $(\mathcal{U}(M), \subseteq)$ is a directed set, this functor is precisely a net of local algebras (note that we have not defined the algebra in a global sense) indexed by $U \in \mathcal{U}(M)$. We close this section with the following definition:

\begin{definition}
    The global algebra $\mathcal{A}(M)$ can be defined by the inductive limit of the directed system
    \begin{equation*}
        \mathcal{A}_\text{alg} : = \left(\bigcup_{U \in \mathcal{U}(M)}\mathcal{A}(U)\right) / \sim \hspace{0.15cm} ,
    \end{equation*}

    where $\sim$ is an equivalence relation generated by the unital injective $^*$-homomorphism $\iota$, i.e., where $A \in \mathcal{A}(U_1)$ and $B \in \mathcal{A}(U_2)$ are identified whenever
    \begin{equation*}
        \iota_{U_1, U_3}(A) = \iota_{U_2, U_3}(B) \hspace{0.15cm} \text{for } U_1, U_2 \subseteq U_3 \hspace{0.15cm} .
    \end{equation*}
    The global algebra is then the completion of $\mathcal{A}_{\text{alg}}$, i.e., $\mathcal{A}(M) =\bar{\mathcal{A}}_{\text{alg}}$.
\end{definition}

Notice that this definition is similar in spirit, for example, to that of a tangent bundle. This is called the inductive limit of the net of local algebras.



\subsection{The Haag-Kastler Axioms:}

It is now time to formally write the Haag-Kastler axioms. These axioms, first developed by Rudolf
Haag and Daniel Kastler \cite{Haag}, are, thus far, the most widely adopted attempt at a rigorous formulation of QFT in Minkowski space. They are the following:

\begin{axiom}(\textbf{Isotony}\footnote{Some authors do not consider this an axiom.})
For $U_1 \subseteq U_2$, the morphism 

\begin{equation*}
    \iota_{U_1, U_2}: \mathcal{A}(U_1) \rightarrow \mathcal{A}(U_2)
\end{equation*}
is a unital $^*$-monomorphism. Thus, we may identify $\mathcal{A}(U_1)$ as the subalgebra of $\mathcal{A}(U_2)$ and write, 

\begin{equation*}
    \mathcal{A}(U_1) \subseteq \mathcal{A}(U_2)\hspace{0.15cm}.
\end{equation*}
    
\end{axiom}


 \begin{axiom} (\textbf{Locality})
    Let $J^\pm$ denote the causal future/past of $U$. Then, given spacelike separated $U_1, U_2 \in \mathcal{U}(M)$, i.e.
     \begin{equation*}
      U_1 \perp U_2 \iff
      J^\pm(U_1) \cap U_2 = \emptyset
      \text{ and }
      J^\pm(U_2) \cap U_1 = \emptyset .
    \end{equation*}
    we have 

    \begin{equation*}
        [A,B] = 0, \hspace{0.15cm} \forall \hspace{0.15cm} A \in \mathcal{A}(U_1), B\in \mathcal{A}(U_2) \hspace{0.15cm} .
    \end{equation*}
\end{axiom}

This reproduces Wightman locality, seen in \textbf{\ref{1.5}}, in an operator-algebraic form: instead of requiring the commutation of field components at spacelike separation, we require that all observables in $\mathcal{A}(U_1)$ commute with all observables in $\mathcal{A}(U_2)$. Recall the linear algebra fact that this means the two algebras are independent, hence the two commuting algebras behave like independent subsystems, exactly as in ordinary quantum mechanics.\\

The next axiom we wish to cover is covariance; therefore, firstly we must demand that the Poincaré group $P$ act on $M$ (more accurately, $\mathcal{U}(M)$). 

\begin{axiom} (\textbf{Covariance})
For each $L \in P$ and $U \in \mathcal{U}(M)$ there exists an isomorphism $\alpha^U_L: \mathcal{A}(U) \rightarrow \mathcal{A}(LU)$, such that for $U_1 \subseteq U_2$

\begin{equation*}
    \alpha_L^{U_2} \circ \iota_{U_1, U_2}= \iota _{LU_1, LU_2} \circ \alpha_L^{U_1}\hspace{0.15cm} ,
\end{equation*}

and $\forall \hspace{0.15cm} L, L^\prime \in P$ and $\forall \hspace{0.15cm} U \in \mathcal{U}(M)$

\begin{equation*}
    \alpha^{LU}_{L^\prime} \circ \alpha^U_L = \alpha^U_{L^\prime L} \hspace{0.15cm} , \hspace{0.25cm} \alpha_{\text{id}}^U = \text{id}_{\mathcal{A}(U)} \hspace{0.15cm} .
\end{equation*}
    
\end{axiom}

This axiom ensures that algebras of regions related via Poincaré transformations are themselves related. Note that, at a first glance, it differs from axiom \textbf{\ref{1.6}},  this is because we have not yet introduced a unitary representation of the Poincaré group. Let $\pi:\mathcal{A}(M) \rightarrow B(\mathcal{H})$ (Bounded operators), be the non-degenerate *-representation
of the global algebra on a Hilbert space \(\mathcal H\). Then, a covariant representation of the net is a pair $(\pi,U)$, where
$U : P \to \mathcal U(\mathcal H)$ is a strongly continuous unitary representation of the
Poincaré group, such that for all $L\in P$, $U\in\mathcal U(M)$,
and $A\in\mathcal A(U)$,
\begin{equation*}
    \pi\big(\alpha_L^U(A)\big) = U(L)\hspace{0.15cm}\pi(A)\hspace{0.15cm}U^{-1}(L).
\end{equation*}
  
In particular, 
\begin{equation*}
 \pi\big(\mathcal A(LU)\big) = U(L)\hspace{0.15cm}\pi\big(\mathcal A(U)\big)\hspace{0.15cm}U^{-1}(L)
  \hspace{0.15cm} .
\end{equation*}

Since $U(L)$ is unitary, $U^{-1}(L) = U^\dagger(L)$. This formulation clearly agrees with the Wightman axioms, with $\pi(A)$ playing the role of the field components. 
















\newpage







\begin{thebibliography}{99}
%1 
\bibitem{dist}
   Petre P. Teodorescu, Wilhelm W. Kecs, and Antonela Toma, \emph{``Distribution Theory''},
   WILEY-VCH Verlag GmbH \& Co., 
   (2013) \href{https://application.wiley-vch.de/books/sample/352741083X_c01.pdf}{[\url{https://application.wiley-vch.de/books/sample/352741083X_c01.pdf}]}.


%2
\bibitem{C^*}
   Ian F. Putnam, \emph{``Lecture notes on $C^*$-algebras''}, 
   (2019) \href{https://web.uvic.ca/~ifputnam/ln/C*-algebras.pdf}{[\url{https://web.uvic.ca/~ifputnam/ln/C*-algebras.pdf}]}.

%3
\bibitem{Haag}
 Haag, Rudolf and Kastler, Daniel, \emph{``An Algebraic approach to quantum field theory''}, J. Math. Phys., 5 (1964).






\end{thebibliography}

\end{document}
